% @Author: WU Zihan
% @Date:   2022-09-11 17:15:57
% @Last Modified by:   WU Zihan
% @Last Modified time: 2022-09-11 17:18:52
\documentclass[10pt, conference, letterpaper]{IEEEtran}
\IEEEoverridecommandlockouts
% The preceding line is only needed to identify funding in the first footnote. If that is unneeded, please comment it out.
\usepackage{cite}
\usepackage{amsmath,amssymb,amsfonts}
%\usepackage{algorithmic}
\usepackage{graphicx}
\usepackage{textcomp}
\usepackage{xcolor}
\usepackage{graphicx}
\usepackage{algorithm}
\usepackage{algorithmicx}
\usepackage{algpseudocode}
\usepackage{caption}
\usepackage{booktabs}
\usepackage{subfigure}
 

\newtheorem{theorem}{Theorem}
\newtheorem{lemma}{Lemma}
\newtheorem{proof}{Proof} 



\begin{document}

\section{Theoretical Analysis}
There are $k$ committees,  and $\mu$ is the maximum transaction processing rate of each committee.

We assume $\lambda$ is the maximum effective throughput. 
On average, there are  $n$ accounts per transaction. $\alpha_{n,m}$ is the probability that a transaction with $n$ accounts involves $m$ shards. 

The overall transaction processing rate of the system is:
\begin{equation} \label{equ1}
	k\mu=\lambda \sum_{m=1}^n m \alpha_{n,m}. 	
\end{equation}


Let $ \chi_i$ be the random variable where
\begin{equation}
	\chi_i=
	\begin{cases} 
		1, & \mbox{if the transaction involves  shard $i$}          \\
		0, & \mbox{if the transaction does not involve   shard $i$}
	\end{cases}
\end{equation}
The probability that any account is assigned to shard $i$ is $\frac{1}{k}$. Correspondingly, the probability that an account is not assigned to $i$ is $1-\frac{1}{k}$. 

Thus, the probability that a transaction does not involve shard $i$ is:
\begin{equation}
	P( \chi_i=0)=\left(1-\frac{1}{k}\right)^n=\left(\frac{k-1}{k}\right)^n.
\end{equation}
Correspondingly, the probability that a transaction involves shard $i$ is:
\begin{equation}
	P( \chi_i=1)=1-P( \chi_i=0)=1-\left(\frac{k-1}{k}\right)^n.
\end{equation}
The expectation is
\begin{equation}
	\begin{split}
		E( \chi_i )&= 0\cdot P( \chi_i=0)+1 \cdot P( \chi_i=1)\\
		&=1-\left(\frac{k-1}{k}\right)^n.
	\end{split}
\end{equation}

Let $\xi_n$ denote the number of shards processing  a transaction with $n$ accounts.		
\begin{equation}
	\xi_n= \chi_1+ \chi_2+\dots+ \chi_k
\end{equation}
According to the linearity of expectation,  
\begin{equation}
	\begin{split}
		E(\xi_n)&=E( \chi_1)+E( \chi_2)+\dots+E( \chi_k)  \\
		&= k\cdot \left[1-\left(\frac{k-1}{k}\right)^n\right]
	\end{split}
\end{equation}
According to the definition of expectation,
\begin{equation}
	E(\xi_n)=\sum_{m=1}^n  \alpha_{n,m}m.
\end{equation}
Thus, 
\begin{equation}
	E(\xi_n)= k\cdot \left[1-\left(\frac{k-1}{k}\right)^n\right]=\sum_{m=1}^n  \alpha_{n,m}m  \\
\end{equation}

Combined with 	Eq.(\ref{equ1}), we can get 
\begin{equation} \label{equ10}
	\begin{split}
		\lambda&=\frac{k\mu}{\sum_{m=1}^n  \alpha_{n,m}m } = \frac{k\mu}{k\cdot \left[1-\left(\frac{k-1}{k}\right)^n\right]}\\
		&=\frac{\mu}{1-\left(\frac{k-1}{k}\right)^n}\\
		&=\frac{\mu k^n}{k^n-(k-1)^n}
	\end{split}
\end{equation}
\begin{equation}  
	\frac{	\partial \lambda}{	\partial k}
	=\frac{\mu n k^{n-1} (k-1)^{n-1} }{\left[k^n-(k-1)^n\right]^2}
	>0
\end{equation}
Eq.(\ref{equ10}) can be transformed into:
\begin{equation}
	\frac{\mu k}{n \lambda}=  \frac{k^n-(k-1)^n}{n k^{n-1}}
\end{equation}

\begin{equation} \label{equ12}
	\begin{split}
		\lim_{k\rightarrow \infty} \frac{\mu k}{n \lambda}&=  \lim_{k\rightarrow \infty} \frac{k^n-(k-1)^n}{n k^{n-1}} \\
		&=\lim_{k\rightarrow \infty}  \frac{k^n-k^n+n k^{n-1}+o(k^{n-1})}{n k^{n-1}} \\
		&= 1
	\end{split}
\end{equation}

Therefore, as $ k \rightarrow \infty$, we can have $\lambda \sim \frac{\mu }{n}k $.



\end{document}