% @Author: WU Zihan
%DIF LATEXDIFF DIFFERENCE FILE
%DIF DEL Theoretical Analysis of Sharding_ori.tex   Sun Sep 11 17:19:12 2022
%DIF ADD Theoretical Analysis of Sharding.tex       Sun Sep 11 17:36:36 2022
% @Date:   2022-09-11 17:15:57
% @Last Modified by:   WU Zihan
%DIF 4c4
%DIF < % @Last Modified time: 2022-09-11 17:46:26
%DIF -------
% @Last Modified time: 2022-09-11 17:34:53
 %DIF > 
%DIF -------
\documentclass[10pt, conference, letterpaper]{IEEEtran}
\IEEEoverridecommandlockouts
% The preceding line is only needed to identify funding in the first footnote. If that is unneeded, please comment it out.
\usepackage{cite}
\usepackage{amsmath,amssymb,amsfonts}
%\usepackage{algorithmic}
\usepackage{graphicx}
\usepackage{textcomp}
\usepackage{xcolor}
\usepackage{graphicx}
\usepackage{algorithm}
\usepackage{algorithmicx}
\usepackage{algpseudocode}
\usepackage{caption}
\usepackage{booktabs}
\usepackage{subfigure}
 

\newtheorem{theorem}{Theorem}
\newtheorem{lemma}{Lemma}
\newtheorem{proof}{Proof} 



%DIF PREAMBLE EXTENSION ADDED BY LATEXDIFF
%DIF UNDERLINE PREAMBLE %DIF PREAMBLE
\RequirePackage[normalem]{ulem} %DIF PREAMBLE
\RequirePackage{color}\definecolor{RED}{rgb}{1,0,0}\definecolor{BLUE}{rgb}{0,0,1} %DIF PREAMBLE
\providecommand{\DIFadd}[1]{{\protect\color{blue}\uwave{#1}}} %DIF PREAMBLE
\providecommand{\DIFdel}[1]{{\protect\color{red}\sout{#1}}}                      %DIF PREAMBLE
%DIF SAFE PREAMBLE %DIF PREAMBLE
\providecommand{\DIFaddbegin}{} %DIF PREAMBLE
\providecommand{\DIFaddend}{} %DIF PREAMBLE
\providecommand{\DIFdelbegin}{} %DIF PREAMBLE
\providecommand{\DIFdelend}{} %DIF PREAMBLE
\providecommand{\DIFmodbegin}{} %DIF PREAMBLE
\providecommand{\DIFmodend}{} %DIF PREAMBLE
%DIF FLOATSAFE PREAMBLE %DIF PREAMBLE
\providecommand{\DIFaddFL}[1]{\DIFadd{#1}} %DIF PREAMBLE
\providecommand{\DIFdelFL}[1]{\DIFdel{#1}} %DIF PREAMBLE
\providecommand{\DIFaddbeginFL}{} %DIF PREAMBLE
\providecommand{\DIFaddendFL}{} %DIF PREAMBLE
\providecommand{\DIFdelbeginFL}{} %DIF PREAMBLE
\providecommand{\DIFdelendFL}{} %DIF PREAMBLE
\newcommand{\DIFscaledelfig}{0.5}
%DIF HIGHLIGHTGRAPHICS PREAMBLE %DIF PREAMBLE
\RequirePackage{settobox} %DIF PREAMBLE
\RequirePackage{letltxmacro} %DIF PREAMBLE
\newsavebox{\DIFdelgraphicsbox} %DIF PREAMBLE
\newlength{\DIFdelgraphicswidth} %DIF PREAMBLE
\newlength{\DIFdelgraphicsheight} %DIF PREAMBLE
% store original definition of \includegraphics %DIF PREAMBLE
\LetLtxMacro{\DIFOincludegraphics}{\includegraphics} %DIF PREAMBLE
\newcommand{\DIFaddincludegraphics}[2][]{{\color{blue}\fbox{\DIFOincludegraphics[#1]{#2}}}} %DIF PREAMBLE
\newcommand{\DIFdelincludegraphics}[2][]{% %DIF PREAMBLE
\sbox{\DIFdelgraphicsbox}{\DIFOincludegraphics[#1]{#2}}% %DIF PREAMBLE
\settoboxwidth{\DIFdelgraphicswidth}{\DIFdelgraphicsbox} %DIF PREAMBLE
\settoboxtotalheight{\DIFdelgraphicsheight}{\DIFdelgraphicsbox} %DIF PREAMBLE
\scalebox{\DIFscaledelfig}{% %DIF PREAMBLE
\parbox[b]{\DIFdelgraphicswidth}{\usebox{\DIFdelgraphicsbox}\\[-\baselineskip] \rule{\DIFdelgraphicswidth}{0em}}\llap{\resizebox{\DIFdelgraphicswidth}{\DIFdelgraphicsheight}{% %DIF PREAMBLE
\setlength{\unitlength}{\DIFdelgraphicswidth}% %DIF PREAMBLE
\begin{picture}(1,1)% %DIF PREAMBLE
\thicklines\linethickness{2pt} %DIF PREAMBLE
{\color[rgb]{1,0,0}\put(0,0){\framebox(1,1){}}}% %DIF PREAMBLE
{\color[rgb]{1,0,0}\put(0,0){\line( 1,1){1}}}% %DIF PREAMBLE
{\color[rgb]{1,0,0}\put(0,1){\line(1,-1){1}}}% %DIF PREAMBLE
\end{picture}% %DIF PREAMBLE
}\hspace*{3pt}}} %DIF PREAMBLE
} %DIF PREAMBLE
\LetLtxMacro{\DIFOaddbegin}{\DIFaddbegin} %DIF PREAMBLE
\LetLtxMacro{\DIFOaddend}{\DIFaddend} %DIF PREAMBLE
\LetLtxMacro{\DIFOdelbegin}{\DIFdelbegin} %DIF PREAMBLE
\LetLtxMacro{\DIFOdelend}{\DIFdelend} %DIF PREAMBLE
\DeclareRobustCommand{\DIFaddbegin}{\DIFOaddbegin \let\includegraphics\DIFaddincludegraphics} %DIF PREAMBLE
\DeclareRobustCommand{\DIFaddend}{\DIFOaddend \let\includegraphics\DIFOincludegraphics} %DIF PREAMBLE
\DeclareRobustCommand{\DIFdelbegin}{\DIFOdelbegin \let\includegraphics\DIFdelincludegraphics} %DIF PREAMBLE
\DeclareRobustCommand{\DIFdelend}{\DIFOaddend \let\includegraphics\DIFOincludegraphics} %DIF PREAMBLE
\LetLtxMacro{\DIFOaddbeginFL}{\DIFaddbeginFL} %DIF PREAMBLE
\LetLtxMacro{\DIFOaddendFL}{\DIFaddendFL} %DIF PREAMBLE
\LetLtxMacro{\DIFOdelbeginFL}{\DIFdelbeginFL} %DIF PREAMBLE
\LetLtxMacro{\DIFOdelendFL}{\DIFdelendFL} %DIF PREAMBLE
\DeclareRobustCommand{\DIFaddbeginFL}{\DIFOaddbeginFL \let\includegraphics\DIFaddincludegraphics} %DIF PREAMBLE
\DeclareRobustCommand{\DIFaddendFL}{\DIFOaddendFL \let\includegraphics\DIFOincludegraphics} %DIF PREAMBLE
\DeclareRobustCommand{\DIFdelbeginFL}{\DIFOdelbeginFL \let\includegraphics\DIFdelincludegraphics} %DIF PREAMBLE
\DeclareRobustCommand{\DIFdelendFL}{\DIFOaddendFL \let\includegraphics\DIFOincludegraphics} %DIF PREAMBLE
%DIF COLORLISTINGS PREAMBLE %DIF PREAMBLE
\RequirePackage{listings} %DIF PREAMBLE
\RequirePackage{color} %DIF PREAMBLE
\lstdefinelanguage{DIFcode}{ %DIF PREAMBLE
%DIF DIFCODE_UNDERLINE %DIF PREAMBLE
  moredelim=[il][\color{red}\sout]{\%DIF\ <\ }, %DIF PREAMBLE
  moredelim=[il][\color{blue}\uwave]{\%DIF\ >\ } %DIF PREAMBLE
} %DIF PREAMBLE
\lstdefinestyle{DIFverbatimstyle}{ %DIF PREAMBLE
	language=DIFcode, %DIF PREAMBLE
	basicstyle=\ttfamily, %DIF PREAMBLE
	columns=fullflexible, %DIF PREAMBLE
	keepspaces=true %DIF PREAMBLE
} %DIF PREAMBLE
\lstnewenvironment{DIFverbatim}{\lstset{style=DIFverbatimstyle}}{} %DIF PREAMBLE
\lstnewenvironment{DIFverbatim*}{\lstset{style=DIFverbatimstyle,showspaces=true}}{} %DIF PREAMBLE
%DIF END PREAMBLE EXTENSION ADDED BY LATEXDIFF

\begin{document}

\section{Theoretical Analysis}
There are $k$ committees,  and $\mu$ is the maximum transaction processing rate of each committee.

We assume $\lambda$ is the maximum effective throughput. 
On average, there are  $n$ accounts per transaction. \DIFdelbegin \DIFdel{$\alpha_{n,m}$ }\DIFdelend \DIFaddbegin \DIFadd{$\alpha_{m}$ }\DIFaddend is the probability that a transaction with $n$ accounts involves $m$ shards. 

The overall transaction processing rate of the system \DIFdelbegin \DIFdel{is:
}\DIFdelend \DIFaddbegin \DIFadd{should satisfy:
}\DIFaddend \begin{equation} \label{equ1}
	\DIFdelbegin \DIFdel{k\mu=}\DIFdelend \lambda \sum_{m=1}^n m \alpha\DIFdelbegin \DIFdel{_{n,m}}\DIFdelend \DIFaddbegin \DIFadd{_{m} }\le \DIFadd{k\mu}\DIFaddend . 	
\end{equation}


Let $ \chi_i$ be the random variable where
\begin{equation}
	\chi_i=
	\begin{cases} 
		1, & \mbox{if the transaction involves  shard $i$}          \\
		0, & \mbox{if the transaction does not involve   shard $i$}
	\end{cases}
\end{equation}
The probability that any account is assigned to shard $i$ is $\frac{1}{k}$. Correspondingly, the probability that an account is not assigned to $i$ is $1-\frac{1}{k}$. 

Thus, the probability that a transaction does not involve shard $i$ is:
\begin{equation}
	\DIFdelbegin \DIFdel{P}\DIFdelend \DIFaddbegin \operatorname{P}\DIFaddend ( \chi_i=0)=\left(1-\frac{1}{k}\right)^n=\left(\frac{k-1}{k}\right)^n.
\end{equation}
Correspondingly, the probability that a transaction involves shard $i$ is:
\begin{equation}
	\DIFdelbegin \DIFdel{P}\DIFdelend \DIFaddbegin \operatorname{P}\DIFaddend ( \chi_i=1)=\DIFdelbegin \DIFdel{1-P}\DIFdelend \DIFaddbegin \DIFadd{1-}\operatorname{P}\DIFaddend ( \chi_i=0)=1-\left(\frac{k-1}{k}\right)^n.
\end{equation}
The expectation is
\begin{equation}
	\DIFdelbegin %DIFDELCMD < \begin{split}
%DIFDELCMD < 		E( \chi_i )&= 0\cdot P( \chi_i=0)+1 \cdot P( \chi_i=1)\\
%DIFDELCMD < 		&=1-\left(\frac{k-1}{k}\right)^n.
%DIFDELCMD < 	\end{split}%%%
\DIFdelend \DIFaddbegin \begin{split}
		\operatorname{E}( \chi_i )&= 0\cdot \operatorname{P}( \chi_i=0)+1 \cdot \operatorname{P}( \chi_i=1)\\
		&=1-\left(\frac{k-1}{k}\right)^n.
	\end{split}\DIFaddend 
\end{equation}

Let \DIFdelbegin \DIFdel{$\xi_n$ }\DIFdelend \DIFaddbegin \DIFadd{$\xi$ }\DIFaddend denote the number of shards processing  a transaction with $n$ accounts.		
\begin{equation}
	\xi\DIFdelbegin \DIFdel{_n}\DIFdelend = \chi_1+ \chi_2+\dots+ \chi_k
\end{equation}
According to the linearity of expectation,  
\begin{equation}
	\DIFdelbegin %DIFDELCMD < \begin{split}
%DIFDELCMD < 		E(\xi_n)&=E( \chi_1)+E( \chi_2)+\dots+E( \chi_k)  \\
%DIFDELCMD < 		&= k\cdot \left[1-\left(\frac{k-1}{k}\right)^n\right]
%DIFDELCMD < 	\end{split}%%%
\DIFdelend \DIFaddbegin \begin{split}
		\operatorname{E}(\xi)&=\operatorname{E}( \chi_1)+\operatorname{E}( \chi_2)+\dots+\operatorname{E}( \chi_k)  \\
		&= k\cdot \left[1-\left(\frac{k-1}{k}\right)^n\right]
	\end{split}\DIFaddend 
\end{equation}
According to the definition of expectation,
\begin{equation}
	\DIFdelbegin \DIFdel{E}\DIFdelend \DIFaddbegin \operatorname{E}\DIFaddend (\xi\DIFdelbegin \DIFdel{_n}\DIFdelend )=\sum_{m=1}^n  \DIFdelbegin \DIFdel{\alpha_{n,m}}\DIFdelend m\DIFaddbegin \DIFadd{\alpha_{m}}\DIFaddend .
\end{equation}
Thus, 
\begin{equation}
	\DIFdelbegin \DIFdel{E(\xi_n)= }\DIFdelend k\cdot \left[1-\left(\frac{k-1}{k}\right)^n\right]=\sum_{m=1}^n  \DIFdelbegin \DIFdel{\alpha_{n,m}}\DIFdelend m\DIFaddbegin \DIFadd{\alpha_{m} = }\operatorname{E}\DIFadd{(\xi) }\DIFaddend \\
\end{equation}

Combined with 	\DIFdelbegin \DIFdel{Eq}\DIFdelend \DIFaddbegin \DIFadd{Ineq}\DIFaddend .(\ref{equ1}), we can get 
\begin{equation} \label{equ10}
	\DIFdelbegin %DIFDELCMD < \begin{split}
%DIFDELCMD < 		\lambda&=\frac{k\mu}{\sum_{m=1}^n  \alpha_{n,m}m } = \frac{k\mu}{k\cdot \left[1-\left(\frac{k-1}{k}\right)^n\right]}\\
%DIFDELCMD < 		&=\frac{\mu}{1-\left(\frac{k-1}{k}\right)^n}\\
%DIFDELCMD < 		&=\frac{\mu k^n}{k^n-(k-1)^n}
%DIFDELCMD < 	\end{split}%%%
\DIFdelend \DIFaddbegin \begin{split}
		\lambda&\le\frac{k\mu}{\sum_{m=1}^n  \alpha_{n,m}m } = \frac{k\mu}{k\cdot \left[1-\left(\frac{k-1}{k}\right)^n\right]}\\
		&=\frac{\mu}{1-\left(\frac{k-1}{k}\right)^n}\\
		&=\frac{\mu k^n}{k^n-(k-1)^n}
	\end{split}\DIFaddend 
\end{equation}
%DIF >  \begin{equation}  
%DIF >  	\frac{	\partial \lambda}{	\partial k}
%DIF >  	=\frac{\mu n k^{n-1} (k-1)^{n-1} }{\left[k^n-(k-1)^n\right]^2}
%DIF >  	>0
%DIF >  \end{equation}
\DIFaddbegin \DIFadd{Or
}\DIFaddend \begin{equation} \DIFdelbegin \DIFdel{\frac{	\partial \lambda}{	\partial k}
	=\frac{\mu n k^{n-1} (k-1)^{n-1} }{\left[k^n-(k-1)^n\right]^2}
	>0
}\DIFdelend \DIFaddbegin \label{equ11}
	\DIFadd{\frac{n\lambda}{\mu k} }\le \DIFadd{\frac{n k^{n-1}}{k^n-(k-1)^n}.
}\DIFaddend \end{equation}
\DIFdelbegin \DIFdel{Eq.(\ref{equ10})
can be transformed into:
}\DIFdelend %DIF >  Eq.(\ref{equ10}) can be transformed into:
%DIF >  \begin{equation}
%DIF >  	\frac{\mu k}{n \lambda}=  \frac{k^n-(k-1)^n}{n k^{n-1}}
%DIF >  \end{equation}
\DIFaddbegin \DIFadd{Consider the right side of Ineq.(\ref{equ11})
}\DIFaddend \begin{equation} \DIFdelbegin \DIFdel{\frac{\mu k}{n \lambda}=  \frac{k^n-(k-1)^n}{n k^{n-1}}
}\DIFdelend \DIFaddbegin \label{equ12}
	\begin{split}
		&\quad \lim_{k\rightarrow \infty} \frac{n k^{n-1}}{k^n-(k-1)^n} \\
		&=\lim_{k\rightarrow \infty}  \frac{n k^{n-1}}{k^n-k^n+n k^{n-1}+o(k^{n-1})} \\
		&= 1
	\end{split}
\DIFaddend \end{equation}

\DIFdelbegin \begin{displaymath} \DIFdel{%DIFDELCMD < \label{equ12}%%%
	\begin{split}
		\lim_{k\rightarrow \infty} \frac{\mu k}{n \lambda}&=  \lim_{k\rightarrow \infty} \frac{k^n-(k-1)^n}{n k^{n-1}} \\
		&=\lim_{k\rightarrow \infty}  \frac{k^n-k^n+n k^{n-1}+o(k^{n-1})}{n k^{n-1}} \\
		&= 1
	\end{split}
}\end{displaymath}%DIFAUXCMD
%DIFDELCMD < 
%DIFDELCMD < 

%DIFDELCMD < %%%
\DIFdelend 

Therefore, as $ k \rightarrow \infty$, we can have \DIFdelbegin \DIFdel{$\lambda \sim \frac{\mu }{n}k $}\DIFdelend \DIFaddbegin \DIFadd{$\lambda \lesssim \cfrac{\mu}{n}\,k $}\DIFaddend .



\end{document}